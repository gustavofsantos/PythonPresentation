\documentclass[aspectratio=169]{beamer}
\usepackage[utf8]{inputenc} % codificacao de caracteres
\usepackage[T1]{fontenc}    % codificacao de fontes
\usepackage[brazil]{babel}  % idioma

\setbeamertemplate{footline}[frame number]
\usetheme{metropolis}         % tema
% \usecolortheme{orchid}      % cores
\usefonttheme[onlymath]{serif} % fonte modo matematico

% Titulo
\title[\sc{Texto no rodap\'e}]{Python, o que é e porque usar}
\author[Gustavo Santos]{Gustavo Fernandes dos Santos \\
\texttt{gfdsantos@inf.ufpel.edu.br}}
\institute{UFPEL - Universidade Federal de Pelotas}

\date{\today}

\begin{document}

\begin{frame}
  \titlepage
\end{frame}

\begin{frame}{Sobre o autor}
    \frametitle{whoami}
    \begin{itemize}
        \item Programador Python há 6 anos
        \item Pregador do software livre há 4 anos
        \item Curioso por natureza
        \item Estudante de Engenharia de Computação
    \end{itemize}
\end{frame}

\begin{frame}{Indice}
    \tableofcontents
\end{frame}

\section{Historinha do Python}

\subsection{Quem usa Python?}

%---------------------------------------------------------

\section{Ferramentas Interessantes}


%---------------------------------------------------------

\section{A Linguagem}


%---------------------------------------------------------

\section{OO e Python 3.x}

\section{Estruturas de Dados em Python}

\section{Hydra - Python Multitarefa}

\section{Coisas Legais no Python}

\section{Referências}

\begin{frame}
    Esta apresentação e todos os scripts apresentados estão disponíveis
no repositório no GitHub: <link do repositorio>
\end{frame}

\end{document}
